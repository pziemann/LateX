\documentclass{article}
\usepackage[polish]{babel}
\usepackage[MeX]{polski}
\usepackage[utf8]{inputenc}
\usepackage[T1]{fontenc}
\usepackage{amsmath}
\usepackage{graphicx}
\title{Zaliczenie z latecha}
\author{Pawel Ziemann}
\begin{document}
\maketitle
\newpage
\tableofcontents
\listoffigures
\section{Tabele}
\subsection{Oceny}
\begin{table}[h]
\begin{tabular}{|c|c|c|}
\hline
Imie & Ocena & ocena2\\ 
\hline 
PUDZIAN & 7 & 3\\ 
\hline 
ADAM MAŁYSZ & 5 & 3\\ 
\hline 
PAPIESZ & 4 & 3\\
\hline 
2137 & 2 & 3\\ 
\hline 
JÓZEF & 4 & 3\\ 
\hline 
\end{tabular}
\end {table}
\subsection{Pi}
\begin{tabular}{c r @{,} l}
Wyrażenie &
\multicolumn{2}{c}{Wartość}\\ \hline %multicolumn, zeby byla liczba znaczaca
$\pi$ & 3&1416 \\
$\pi^{\pi}$ & 36&46 \\
$(\pi^{\pi})^{\pi}$ & 80662&7 \\
\end{tabular}
\section{Wzory matematyczne}
\subsection{Heron}
\begin{center}
{\LARGE Wzór Herona }
\end{center}
\begin{equation}
\mathit{P_\Delta} = \sqrt{p(p-a)(p-b)(p-c)}
\end{equation}
\section{Itemizacja}
\begin{itemize}
\item KEKW
\item Kappa
\item DansGame
\item PepooLove
\end{itemize}
\begin{figure}
\includegraphics[scale=0.2]{benc.jpg}
\caption{Heron, koloryzowane}
\end{figure}
\begin{thebibliography}{9}
\bibitem{1}
 Paweł Ziemann,
 \emph{Coś ważnego}.
 Publikacja 1,
 2020.
 \bibitem{2}
 też coś
\end{thebibliography}
\end{document}